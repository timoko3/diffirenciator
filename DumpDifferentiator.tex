\documentclass[a4paper,12pt]{article}
\usepackage[a4paper,top=1.3cm,bottom=2cm,left=1.5cm,right=1.5cm,marginparwidth=0.75cm]{geometry}
\usepackage[T2A]{fontenc}
\usepackage[utf8]{inputenc}
\usepackage[english,russian]{babel}
\usepackage{graphicx}
\usepackage{breqn}\usepackage{pgfplots}

\pgfplotsset{compat=1.18}

\newcommand{\plus}[2]{#1 + #2}
\newcommand{\minus}[2]{#1 - #2}
\newcommand{\mul}[2]{#1 * #2}
\newcommand{\pow}[2]{#1 ^ #2}
\begin{document}
\thispagestyle{empty}
\newgeometry{left=0cm,right=0cm,top=0cm,bottom=0cm}
\begin{figure}
\centering
\includegraphics[width=\paperwidth,height=\paperheight,keepaspectratio]{matan.jpg}
\caption{Параметры системы во время исполнения теста}
\end{figure}
\restoregeometry
\section{Предостережения}
\begin{itemize}
\item (ГЛАВНАЯ АКСИОМА МАТАНА) Помни, что если не продифференцируешь ты, то продиффиринцируют тебя!!
\item Никаких действительных чисел! Мне хватило сечений Дедекинда. Ради вашего же блага прошу вводить только целые. При попытке использовать другие числа я не могу ручаться за сохранность французского на вашем компьютере!!!
\end{itemize}
\section{Постановка задачи}
Возьмем производную данного выражения:
\begin{dmath}
 y(x)=\plus{\sin{(\plus{\mul{15}{\pow{x}{3}}}{7})}}{\pow{\cos{(\plus{\mul{9}{x}}{6})}}{5}}\end{dmath}
\section{Непосредственно (прости господи) дифференцирование и элементарное упрощение выражения}
Данные преобразования поистенне чудесны, но у меня не хватает свободного пространство, чтобы показать это. Пусть это останется несложным упражнением для читателя.

\begin{dmath}
 y'(x)=\plus{\mul{(\cos{(\plus{\mul{15}{\pow{x}{3}}}{7})})}{(\plus{\plus{\mul{0}{\pow{x}{3}}}{\mul{15}{\mul{1}{\mul{3}{\pow{x}{\minus{3}{1}}}}}}}{0})}}{\mul{(\mul{(\mul{\sin{(\plus{\mul{9}{x}}{6})}}{-1})}{(\plus{\plus{\mul{0}{x}}{\mul{9}{1}}}{0})})}{(\mul{5}{\pow{\cos{(\plus{\mul{9}{x}}{6})}}{\minus{5}{1}}})}}\end{dmath}
\begin{dmath}
 y'(x)=\plus{\mul{(\cos{(\plus{\mul{15}{\pow{x}{3}}}{7})})}{(\plus{\plus{0}{\mul{15}{\mul{1}{\mul{3}{\pow{x}{\minus{3}{1}}}}}}}{0})}}{\mul{(\mul{(\mul{\sin{(\plus{\mul{9}{x}}{6})}}{-1})}{(\plus{\plus{\mul{0}{x}}{\mul{9}{1}}}{0})})}{(\mul{5}{\pow{\cos{(\plus{\mul{9}{x}}{6})}}{\minus{5}{1}}})}}\end{dmath}
Т.к. 1+0=1, следовательно 0+1=1. Дальнейшие преобразования элементарны, поэтому не буду утруждать вас их чтением

\begin{dmath}
 y'(x)=\plus{\mul{(\cos{(\plus{\mul{15}{\pow{x}{3}}}{7})})}{(\plus{\plus{0}{\mul{15}{\mul{1}{\mul{3}{\pow{x}{2}}}}}}{0})}}{\mul{(\mul{(\mul{\sin{(\plus{\mul{9}{x}}{6})}}{-1})}{(\plus{\plus{\mul{0}{x}}{\mul{9}{1}}}{0})})}{(\mul{5}{\pow{\cos{(\plus{\mul{9}{x}}{6})}}{\minus{5}{1}}})}}\end{dmath}
\begin{dmath}
 y'(x)=\plus{\mul{(\cos{(\plus{\mul{15}{\pow{x}{3}}}{7})})}{(\plus{\plus{0}{\mul{15}{\mul{3}{\pow{x}{2}}}}}{0})}}{\mul{(\mul{(\mul{\sin{(\plus{\mul{9}{x}}{6})}}{-1})}{(\plus{\plus{\mul{0}{x}}{\mul{9}{1}}}{0})})}{(\mul{5}{\pow{\cos{(\plus{\mul{9}{x}}{6})}}{\minus{5}{1}}})}}\end{dmath}
Здесь все тривиальнейшим образом сокращается

\begin{dmath}
 y'(x)=\plus{\mul{(\cos{(\plus{\mul{15}{\pow{x}{3}}}{7})})}{(\plus{\mul{15}{\mul{3}{\pow{x}{2}}}}{0})}}{\mul{(\mul{(\mul{\sin{(\plus{\mul{9}{x}}{6})}}{-1})}{(\plus{\plus{\mul{0}{x}}{\mul{9}{1}}}{0})})}{(\mul{5}{\pow{\cos{(\plus{\mul{9}{x}}{6})}}{\minus{5}{1}}})}}\end{dmath}
\begin{dmath}
 y'(x)=\plus{\mul{(\cos{(\plus{\mul{15}{\pow{x}{3}}}{7})})}{(\mul{15}{\mul{3}{\pow{x}{2}}})}}{\mul{(\mul{(\mul{\sin{(\plus{\mul{9}{x}}{6})}}{-1})}{(\plus{\plus{\mul{0}{x}}{\mul{9}{1}}}{0})})}{(\mul{5}{\pow{\cos{(\plus{\mul{9}{x}}{6})}}{\minus{5}{1}}})}}\end{dmath}
Обладая базовыми знаниями матеметики нетрудно заметить, что...

\begin{dmath}
 y'(x)=\plus{\mul{(\cos{(\plus{\mul{15}{\pow{x}{3}}}{7})})}{(\mul{15}{\mul{3}{\pow{x}{2}}})}}{\mul{(\mul{(\mul{\sin{(\plus{\mul{9}{x}}{6})}}{-1})}{(\plus{\plus{0}{\mul{9}{1}}}{0})})}{(\mul{5}{\pow{\cos{(\plus{\mul{9}{x}}{6})}}{\minus{5}{1}}})}}\end{dmath}
\begin{dmath}
 y'(x)=\plus{\mul{(\cos{(\plus{\mul{15}{\pow{x}{3}}}{7})})}{(\mul{15}{\mul{3}{\pow{x}{2}}})}}{\mul{(\mul{(\mul{\sin{(\plus{\mul{9}{x}}{6})}}{-1})}{(\plus{\plus{0}{9}}{0})})}{(\mul{5}{\pow{\cos{(\plus{\mul{9}{x}}{6})}}{\minus{5}{1}}})}}\end{dmath}
Обладая базовыми знаниями матеметики нетрудно заметить, что...

\begin{dmath}
 y'(x)=\plus{\mul{(\cos{(\plus{\mul{15}{\pow{x}{3}}}{7})})}{(\mul{15}{\mul{3}{\pow{x}{2}}})}}{\mul{(\mul{(\mul{\sin{(\plus{\mul{9}{x}}{6})}}{-1})}{(\plus{9}{0})})}{(\mul{5}{\pow{\cos{(\plus{\mul{9}{x}}{6})}}{\minus{5}{1}}})}}\end{dmath}
\begin{dmath}
 y'(x)=\plus{\mul{(\cos{(\plus{\mul{15}{\pow{x}{3}}}{7})})}{(\mul{15}{\mul{3}{\pow{x}{2}}})}}{\mul{(\mul{\mul{\sin{(\plus{\mul{9}{x}}{6})}}{-1}}{9})}{(\mul{5}{\pow{\cos{(\plus{\mul{9}{x}}{6})}}{\minus{5}{1}}})}}\end{dmath}
Для того, чтобы точно понять данное преобразование советую обратиться к пособию Саблезубова Петра Ивановича к тому 5 сочинений по теме "1000 способов вскрыть черепную коробку при помощи интеграла" страница 666 3 абзац формула (17.2) для точного ее понимаю желательно прочитать предыдущие 3 параграфа

\begin{dmath}
 y'(x)=\plus{\mul{(\cos{(\plus{\mul{15}{\pow{x}{3}}}{7})})}{(\mul{15}{\mul{3}{\pow{x}{2}}})}}{\mul{(\mul{\mul{\sin{(\plus{\mul{9}{x}}{6})}}{-1}}{9})}{(\mul{5}{\pow{\cos{(\plus{\mul{9}{x}}{6})}}{4}})}}\end{dmath}
\section{И снова Тейлор ...- - -...}
Расчленим исходную функцию до $o(x^0)$.
\begin{dmath}
 y'(x)=\mul{(\frac{\plus{\sin{(\plus{\mul{15}{\pow{x}{3}}}{7})}}{\pow{\cos{(\plus{\mul{9}{x}}{6})}}{5}}}{1})}{(\pow{x}{0})}+o(x^0)
\end{dmath}
Объяснять не буду т.к. если вам это не очевидно, значит вам пора перестать заниматься математикой.

\begin{dmath}
 y'(x)=\mul{(\plus{\sin{(\plus{\mul{15}{\pow{x}{3}}}{7})}}{\pow{\cos{(\plus{\mul{9}{x}}{6})}}{5}})}{(\pow{x}{0})}\end{dmath}
\begin{dmath}
 y'(x)=\mul{(\plus{\sin{(\plus{\mul{15}{\pow{x}{3}}}{7})}}{\pow{\cos{(\plus{\mul{9}{x}}{6})}}{5}})}{(\pow{x}{0})}+o(x^0)
\end{dmath}
\section{Результат вычислений:}
Вот мы и получили результат, который с легкостью получал любой пятиклассник в СССР быстрее данной программы. Живите с этим.
\begin{dmath}
y'(x)=\plus{\mul{(\cos{(\plus{\mul{15}{\pow{x}{3}}}{7})})}{(\mul{15}{\mul{3}{\pow{x}{2}}})}}{\mul{(\mul{\mul{\sin{(\plus{\mul{9}{x}}{6})}}{-1}}{9})}{(\mul{5}{\pow{\cos{(\plus{\mul{9}{x}}{6})}}{4}})}}\end{dmath}
\section{Теперь, чтобы все стало совсем понятно(ахахахахаахахаххаха).\\Построим график полученной производной исходной функции}
\begin{center}
\begin{tikzpicture}
\begin{axis}[
width=16cm,
height=8cm,
domain=-10:10,
samples=200,
axis lines=middle,
xlabel={$x$},
ylabel={$y$},
grid=both]
\addplot[thick, red] {((cos(((15)*((x)^(3)))+(7)))*((15)*((3)*((x)^(2)))))+((((sin(((9)*(x))+(6)))*(-1))*(9))*((5)*((cos(((9)*(x))+(6)))^(4))))};
\end{axis}
\end{tikzpicture}
\end{center}

\section{P.S}Уважаемая кафедра высшей математики не принимайте всерьез данную работу. Автор на самом деле очень любит матан. Все персонажи вымышлены(почти) и ни один учебник математики не пострадал.\end{document}
