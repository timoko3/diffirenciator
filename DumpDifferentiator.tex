\documentclass[a4paper,12pt]{article}
\usepackage[a4paper,top=1.3cm,bottom=2cm,left=1.5cm,right=1.5cm,marginparwidth=0.75cm]{geometry}
\usepackage[T2A]{fontenc}
\usepackage[utf8]{inputenc}
\usepackage[english,russian]{babel}
\usepackage{graphicx}
\usepackage{pgfplots}

\pgfplotsset{compat=1.18}

\newcommand{\plus}[2]{#1 + #2}
\newcommand{\minus}[2]{#1 - #2}
\newcommand{\mul}[2]{#1 * #2}
\begin{document}
\thispagestyle{empty}
\newgeometry{left=0cm,right=0cm,top=0cm,bottom=0cm}
\begin{figure}
\centering
\includegraphics[width=\paperwidth,height=\paperheight,keepaspectratio]{matan.jpg}
\caption{Параметры системы во время исполнения теста}
\end{figure}
\restoregeometry
\section{Предостережение}
Никаких действительных чисел! Мне хватило сечений Дедекинда. Ради вашего же блага прошу вводить только целые числа. При попытке использовать другие числа я не могу ручаться за сохранность французского на вашем компьютере
\section{Постановка задачи}
Возьмем производную данного выражения:
\[
\cos{(\plus{(x)}{(x)})}\]
\section{Непосредственно (прости господи) дифференцирование и элементарное упрощение выражения}
\[
\mul{(\mul{(\sin{(\plus{(x)}{(x)})})}{(-1)})}{(\plus{(1)}{(1)})}\]
\[
\mul{(\mul{(\sin{(\plus{(x)}{(x)})})}{(-1)})}{(\plus{(1)}{(1)})}\]
\[
\mul{(\mul{(\sin{(\plus{(x)}{(x)})})}{(-1)})}{(\plus{(1)}{(1)})}\]
\[
\mul{(\mul{(\sin{(\plus{(x)}{(x)})})}{(-1)})}{(\plus{(1)}{(1)})}\]
Данное преобразование поистенне чудесно, но у меня не хватает свободного пространство, чтобы показать это. Пусть это останется несложным упражнением для читателя.

\[
\mul{(\mul{(\sin{(\plus{(x)}{(x)})})}{(-1)})}{(\plus{(1)}{(1)})}\]
\[
\mul{(\mul{(\sin{(\plus{(x)}{(x)})})}{(-1)})}{(\plus{(1)}{(1)})}\]
\[
\mul{(\mul{(\sin{(\plus{(x)}{(x)})})}{(-1)})}{(\plus{(1)}{(1)})}\]
\[
\mul{(\mul{(\sin{(\plus{(x)}{(x)})})}{(-1)})}{(\plus{(1)}{(1)})}\]
\[
\mul{(\mul{(\sin{(\plus{(x)}{(x)})})}{(-1)})}{(\plus{(1)}{(1)})}\]
\[
\mul{(\mul{(\sin{(\plus{(x)}{(x)})})}{(-1)})}{(\plus{(1)}{(1)})}\]
Здесь все тривиальнейшим образом сокращается

\[
\mul{(\mul{(\sin{(\plus{(x)}{(x)})})}{(-1)})}{(\plus{(1)}{(1)})}\]
\[
\mul{(\mul{(\sin{(\plus{(x)}{(x)})})}{(-1)})}{(\plus{(1)}{(1)})}\]
\[
\mul{(\mul{(\sin{(\plus{(x)}{(x)})})}{(-1)})}{(2)}\]
\[
\mul{(\mul{(\sin{(\plus{(x)}{(x)})})}{(-1)})}{(2)}\]
\[
\mul{(\mul{(\sin{(\plus{(x)}{(x)})})}{(-1)})}{(2)}\]
\[
\mul{(\mul{(\sin{(\plus{(x)}{(x)})})}{(-1)})}{(2)}\]
\section{Теперь, чтобы все стало совсем понятно(ахахахахаахахаххаха).\\Построим график полученной производной исходной функции}
\begin{tikzpicture}
\begin{axis}[
width=16cm,
height=8cm,
domain=-100:100,
samples=200,
axis lines=middle,
xlabel={$x$},
ylabel={$y$},
grid=both]
\addplot[thick, red] {((sin((x)+(x)))*(-1))*(2)};
\end{axis}
\end{tikzpicture}

\end{document}
