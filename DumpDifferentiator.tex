\documentclass[a4paper,12pt]{article}
\usepackage[a4paper,top=1.3cm,bottom=2cm,left=1.5cm,right=1.5cm,marginparwidth=0.75cm]{geometry}
\usepackage[T2A]{fontenc}
\usepackage[utf8]{inputenc}
\usepackage[english,russian]{babel}
\usepackage{graphicx}
\usepackage{pgfplots}

\pgfplotsset{compat=1.18}

\newcommand{\plus}[2]{#1 + #2}
\newcommand{\minus}[2]{#1 - #2}
\newcommand{\mul}[2]{#1 * #2}
\newcommand{\pow}[2]{#1 ^ #2}
\begin{document}
\thispagestyle{empty}
\newgeometry{left=0cm,right=0cm,top=0cm,bottom=0cm}
\begin{figure}
\centering
\includegraphics[width=\paperwidth,height=\paperheight,keepaspectratio]{matan.jpg}
\caption{Параметры системы во время исполнения теста}
\end{figure}
\restoregeometry
\section{Предостережения}
\begin{itemize}
\item (ГЛАВНАЯ АКСИОМА МАТАНА) Помни, что если не продифференцируешь ты, то продиффиринцируют тебя!!
\item Никаких действительных чисел! Мне хватило сечений Дедекинда. Ради вашего же блага прошу вводить только целые числа. При попытке использовать другие числа я не могу ручаться за сохранность французского на вашем компьютере!!!
\end{itemize}
\section{Постановка задачи}
Возьмем производную данного выражения:
\[
 y(x)=\pow{x}{2}\]
\section{Непосредственно (прости господи) дифференцирование и элементарное упрощение выражения}
\[
 y'(x)=\mul{(\pow{x}{2})}{(\plus{\mul{(0)}{(\ln{(x)})}}{\mul{(\frac{(1)}{(x)})}{(2)}})}\]
\[
 y'(x)=\mul{(\pow{x}{2})}{(\plus{\mul{(0)}{(\ln{(x)})}}{\mul{(\frac{(1)}{(x)})}{(2)}})}\]
\[
 y'(x)=\mul{(\pow{x}{2})}{(\plus{\mul{(0)}{(\ln{(x)})}}{\mul{(\frac{(1)}{(x)})}{(2)}})}\]
\[
 y'(x)=\mul{(\pow{x}{2})}{(\plus{\mul{(0)}{(\ln{(x)})}}{\mul{(\frac{(1)}{(x)})}{(2)}})}\]
Здесь все тривиальнейшим образом сокращается

\[
 y'(x)=\mul{(\pow{x}{2})}{(\plus{\mul{(0)}{(\ln{(x)})}}{\mul{(\frac{(1)}{(x)})}{(2)}})}\]
\[
 y'(x)=\mul{(\pow{x}{2})}{(\plus{\mul{(0)}{(\ln{(x)})}}{\mul{(\frac{(1)}{(x)})}{(2)}})}\]
\[
 y'(x)=\mul{(\pow{x}{2})}{(\plus{\mul{(0)}{(\ln{(x)})}}{\mul{(\frac{(1)}{(x)})}{(2)}})}\]
\[
 y'(x)=\mul{(\pow{x}{2})}{(\plus{\mul{(0)}{(\ln{(x)})}}{\mul{(\frac{(1)}{(x)})}{(2)}})}\]
\[
 y'(x)=\mul{(\pow{x}{2})}{(\plus{\mul{(0)}{(\ln{(x)})}}{\mul{(\frac{(1)}{(x)})}{(2)}})}\]
\[
 y'(x)=\mul{(\pow{x}{2})}{(\plus{0}{\mul{(\frac{(1)}{(x)})}{(2)}})}\]
Объяснять не буду т.к. если вам это не очевидно, значит вам пора перестать заниматься математикой.

\[
 y'(x)=\mul{(\pow{x}{2})}{(\plus{0}{\mul{(\frac{(1)}{(x)})}{(2)}})}\]
\[
 y'(x)=\mul{(\pow{x}{2})}{(\plus{0}{\mul{(\frac{(1)}{(x)})}{(2)}})}\]
\[
 y'(x)=\mul{(\pow{x}{2})}{(\plus{0}{\mul{(\frac{(1)}{(x)})}{(2)}})}\]
\[
 y'(x)=\mul{(\pow{x}{2})}{(\plus{0}{\mul{(\frac{(1)}{(x)})}{(2)}})}\]
\[
 y'(x)=\mul{(\pow{x}{2})}{(\plus{0}{\mul{(\frac{(1)}{(x)})}{(2)}})}\]
\[
 y'(x)=\mul{(\pow{x}{2})}{(\plus{0}{\mul{(\frac{(1)}{(x)})}{(2)}})}\]
Здесь все тривиальнейшим образом сокращается

\[
 y'(x)=\mul{(\pow{x}{2})}{(\plus{0}{\mul{(\frac{(1)}{(x)})}{(2)}})}\]
\[
 y'(x)=\mul{(\pow{x}{2})}{(\plus{0}{\mul{(\frac{(1)}{(x)})}{(2)}})}\]
\[
 y'(x)=\mul{(\pow{x}{2})}{(\plus{0}{\mul{(\frac{(1)}{(x)})}{(2)}})}\]
\[
 y'(x)=\mul{(\pow{x}{2})}{(\plus{0}{\mul{(\frac{(1)}{(x)})}{(2)}})}\]
\section{Результат вычислений:}
\[
y'(x)=\mul{(\pow{x}{2})}{(\plus{0}{\mul{(\frac{(1)}{(x)})}{(2)}})}\]
\section{Теперь, чтобы все стало совсем понятно(ахахахахаахахаххаха).\\Построим график полученной производной исходной функции}
\centering\begin{tikzpicture}
\begin{axis}[
width=16cm,
height=8cm,
domain=-100:100,
samples=200,
axis lines=middle,
xlabel={$x$},
ylabel={$y$},
grid=both]
\addplot[thick, red] {((x)^(2))*((0)+(((1)/(x))*(2)))};
\end{axis}
\end{tikzpicture}

\end{document}
