\documentclass[a4paper,12pt]{article}
\usepackage[a4paper,top=1.3cm,bottom=2cm,left=1.5cm,right=1.5cm,marginparwidth=0.75cm]{geometry}
\usepackage[T2A]{fontenc}
\usepackage[utf8]{inputenc}
\usepackage[english,russian]{babel}
\usepackage{graphicx}
\usepackage{pgfplots}

\pgfplotsset{compat=1.18}

\newcommand{\plus}[2]{#1 + #2}
\newcommand{\minus}[2]{#1 - #2}
\newcommand{\mul}[2]{#1 * #2}
\newcommand{\pow}[2]{#1 ^ #2}
\begin{document}
\thispagestyle{empty}
\newgeometry{left=0cm,right=0cm,top=0cm,bottom=0cm}
\begin{figure}
\centering
\includegraphics[width=\paperwidth,height=\paperheight,keepaspectratio]{matan.jpg}
\caption{Параметры системы во время исполнения теста}
\end{figure}
\restoregeometry
\section{Предостережения}
\begin{itemize}
\item (ГЛАВНАЯ АКСИОМА МАТАНА) Помни, что если не продифференцируешь ты, то продиффиринцируют тебя!!
\item Никаких действительных чисел! Мне хватило сечений Дедекинда. Ради вашего же блага прошу вводить только целые числа. При попытке использовать другие числа я не могу ручаться за сохранность французского на вашем компьютере!!!
\end{itemize}
\section{Постановка задачи}
Возьмем производную данного выражения:
\[
 y(x)=\sin{(x)}\]
\section{Непосредственно (прости господи) дифференцирование и элементарное упрощение выражения}
\[
 y'(x)=\mul{(\cos{(x)})}{(1)}\]
\[
 y'(x)=\mul{(\cos{(x)})}{(1)}\]
\[
 y'(x)=\mul{(\cos{(x)})}{(1)}\]
\[
 y'(x)=\mul{(\cos{(x)})}{(1)}\]
Для того, чтобы точно понять данное преобразование советую обратиться к пособию Саблезубова Акакия Акакиявеча к тому 5 сочинений по теме "1000 способов вскрыть черепную коробку при помощи интеграла" страница 666 3 абзац формула (17.2) для точного ее понимаю желательно прочитать предыдущие 3 параграфа

\[
 y'(x)=\mul{(\cos{(x)})}{(1)}\]
\[
 y'(x)=\mul{(\cos{(x)})}{(1)}\]
\[
 y'(x)=\mul{(\cos{(x)})}{(1)}\]
\section{Тейлор...}\[
 y'(x)=\plus{\plus{\plus{\mul{(\frac{(\sin{(x)})}{(1)})}{(\pow{x}{0})}}}{\mul{(\frac{(\mul{(\cos{(x)})}{(1)})}{(1)})}{(\pow{x}{1})}}}{\mul{(\frac{(\plus{\mul{(\mul{(\mul{(\sin{(x)})}{(-1)})}{(1)})}{(1)}}{\mul{(\cos{(x)})}{(0)}})}{(1)})}{(\pow{x}{2})}}\]
\section{Результат вычислений:}
\[
y'(x)=\mul{(\cos{(x)})}{(1)}\]
\section{Теперь, чтобы все стало совсем понятно(ахахахахаахахаххаха).\\Построим график полученной производной исходной функции}
\centering\begin{tikzpicture}
\begin{axis}[
width=16cm,
height=8cm,
domain=-10:10,
samples=200,
axis lines=middle,
xlabel={$x$},
ylabel={$y$},
grid=both]
\addplot[thick, red] {(cos(x))*(1)};
\end{axis}
\end{tikzpicture}

\end{document}
